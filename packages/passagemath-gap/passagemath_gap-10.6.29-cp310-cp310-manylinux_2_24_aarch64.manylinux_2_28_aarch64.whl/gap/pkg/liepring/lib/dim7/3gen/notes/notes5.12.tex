
\documentclass[12pt]{article}
%%%%%%%%%%%%%%%%%%%%%%%%%%%%%%%%%%%%%%%%%%%%%%%%%%%%%%%%%%%%%%%%%%%%%%%%%%%%%%%%%%%%%%%%%%%%%%%%%%%%%%%%%%%%%%%%%%%%%%%%%%%%%%%%%%%%%%%%%%%%%%%%%%%%%%%%%%%%%%%%%%%%%%%%%%%%%%%%%%%%%%%%%%%%%%%%%%%%%%%%%%%%%%%%%%%%%%%%%%%%%%%%%%%%%%%%%%%%%%%%%%%%%%%%%%%%
\usepackage{amsfonts}
\usepackage{amssymb}
\usepackage{sw20elba}

%TCIDATA{OutputFilter=LATEX.DLL}
%TCIDATA{Version=5.50.0.2890}
%TCIDATA{<META NAME="SaveForMode" CONTENT="1">}
%TCIDATA{BibliographyScheme=Manual}
%TCIDATA{Created=Monday, July 01, 2013 14:26:16}
%TCIDATA{LastRevised=Tuesday, July 02, 2013 14:53:38}
%TCIDATA{<META NAME="GraphicsSave" CONTENT="32">}
%TCIDATA{<META NAME="DocumentShell" CONTENT="Articles\SW\mrvl">}
%TCIDATA{CSTFile=LaTeX article (bright).cst}
%TCIDATA{ComputeDefs=
%$\varepsilon =1$
%}


\newtheorem{theorem}{Theorem}
\newtheorem{axiom}[theorem]{Axiom}
\newtheorem{claim}[theorem]{Claim}
\newtheorem{conjecture}[theorem]{Conjecture}
\newtheorem{corollary}[theorem]{Corollary}
\newtheorem{definition}[theorem]{Definition}
\newtheorem{example}[theorem]{Example}
\newtheorem{exercise}[theorem]{Exercise}
\newtheorem{lemma}[theorem]{Lemma}
\newtheorem{notation}[theorem]{Notation}
\newtheorem{problem}[theorem]{Problem}
\newtheorem{proposition}[theorem]{Proposition}
\newtheorem{remark}[theorem]{Remark}
\newtheorem{solution}[theorem]{Solution}
\newtheorem{summary}[theorem]{Summary}
\newenvironment{proof}[1][Proof]{\noindent\textbf{#1.} }{{\hfill $\Box$ \\}}
\input{tcilatex}
\addtolength{\textheight}{30pt}

\begin{document}

\title{Descendants of algebra 5.12 of order $p^{7}$}
\author{Michael Vaughan-Lee}
\date{July 2013}
\maketitle

We have two four parameter families of descendants of algebra 5.12 of order $%
p^{7}$. The parameters are $x,y,z,t$ in both cases.

\section{Note 1}

We put the parameters $x,y,z,t$ in a matrix $\left( 
\begin{array}{ll}
x & y \\ 
z & t%
\end{array}%
\right) $, and the distinct algebras correspond to orbits of matrices $%
A=\left( 
\begin{array}{ll}
x & y \\ 
z & t%
\end{array}%
\right) $ with entries in GF$(p)$ under the action%
\[
A\rightarrow \frac{1}{\det P}PAP^{-1}
\]%
where $P$ is the subgroup of GL$(2,p)$ consisting of non-singular matrices $%
\left( 
\begin{array}{ll}
\alpha  & \beta  \\ 
\beta  & \alpha 
\end{array}%
\right) $ or $\left( 
\begin{array}{ll}
\alpha  & \beta  \\ 
-\beta  & -\alpha 
\end{array}%
\right) $. So we want to pick out a set of orbit representatives.
Notes5.12.m is a \textsc{Magma} program which outputs a matrix mats1 with
suitable $[x,y,z,t]$ as rows.

\section{Note 2}

We put the parameters $x,y,z,t$ in a matrix $\left( 
\begin{array}{ll}
x & y \\ 
z & t%
\end{array}%
\right) $, and the distinct algebras correspond to orbits of matrices $%
A=\left( 
\begin{array}{ll}
x & y \\ 
z & t%
\end{array}%
\right) $ with entries in GF$(p)$ under the action%
\[
A\rightarrow \frac{1}{\det P}PAP^{-1}
\]%
where $P$ is the subgroup of GL$(2,p)$ consisting of non-singular matrices $%
\left( 
\begin{array}{ll}
\alpha  & \omega \beta  \\ 
\beta  & \alpha 
\end{array}%
\right) $ or $\left( 
\begin{array}{ll}
\alpha  & \omega \beta  \\ 
-\beta  & -\alpha 
\end{array}%
\right) $. So we want to pick out a set of orbit representatives.
Notes5.12.m is a \textsc{Magma} program which outputs a matrix mats2 with
suitable $[x,y,z,t]$ as rows.

\end{document}
